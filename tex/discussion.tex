\section{Discussion}
\label{sec:discussion}
By comparing the learning curves in Figure~\ref{fig:learning_curves}, all models converge after 150 epochs and starts overfitting after around 20 epochs of training even with regularization. The regularization does seem to lower overfitting somewhat, when validation curves move up and training curves down in accuracy, like e.g.\ for model 2 with $L_2$ regularization and dropout, where the overall span and area between the (purple) curves is lower. So model 2 seems to generalize better than model 1, but the difference is so marginal, that they could almost be the same model.

As mentioned in the experimental setup, model 2 could be forced to not reducing into model 1, by only providing the previous output as input, but following the test results for model 3 in Table~\ref{tab:test_eval} this seems to handicap model 2 a lot.  

To validate whether gradients vanish through backpropagation, the weights are investigated in Appendix~\ref{sec:gru_weights}, where it is seen that the frobenius norm of the weight matrices keep on increasing and the mean varying gradually, so the weights are adjusted during training. 

The model 1 and 2 reconstructions can be compared to the original example in Figure~\ref{fig:reconstructions}, where it is seen that the models can definitely catch the scale of the melody and mostly predict pitches belonging to that. Some jumps in pitch (intervals) are predicted correctly.

Predicting durations should be much easier for the models and they seem to be produced with the same frequency as in the original melody, but mostly misplaced. This is a mere artifact of the feature formatting itself, as one wrong prediction of a duration will shift the whole melody and not align with the original melody. This could be solved by using continous relative positions instead of duration classes. 

By ignoring the duration shifts in the reconstructions, some rhythmical motifs becomes clear. Some of the jumps between durations of $1/4$ (notes with no flag) and $1/8$ (notes with one flag) are reconstructed correctly by model 2 in bar 3, where the sequence is ($1/4$, $1/8$, $1/4$, $1/8$). 
