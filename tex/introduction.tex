\section{Introduction}
\label{sec:introduction}

\begin{description}
	\item[Objective] We want to make a generative model for music notation using a recurrent neural network, inputting just a few notes, and getting a melody out.
	
	\item[Motivation] Several:
	
	\begin{itemize}
		\item We wanted to work with RNNs.
		
		\item Our original plan was to work with electronic healthcare records (EHR), but the sequences were too short (order of unity).
		
		\item Music has rules for what sounds good: notes that go together (scales), repeating rhythms.
		
		\item We are both musicians (to greater or lesser degrees).
	\end{itemize}
	
	\item[Proposal] We propose two models using RNNs using different structure.
	
	\item[Previous work] Using same data as us.
	
	\begin{description}
		\item[Sturm] Uses character RNN from Andrey Karpathy to generate musical notation encoded in text format, getting reasonable results after correcting the generated codes.
		
		```Lisl's Stis': Recurrent Neural Networks for Folk Music Generation''
		Bob L. Sturm (blog post from May 2015)
		
		\item[Zimmerman] Uses melody (pitch at every 16th beat) and harmony (chord at every 16th beat), getting accuracies of around 77 \%. Same network structure as ours.
		
		``A Dual Classification Approach to Music Language Modelling''
		Zimmerman (2016)
	\end{description}
\end{description}

\subsection*{EHR synopsis}

Forecasting a patients health history from previous diagnoses is of huge interest for the medical science, as further development of diseases and complications can be avoided by choosing the right treatment in the right time. Today’s health registry databases are vast and still evolving, so models analysing all patients’s health history as a whole could help determine disease trajectories on a global scale.

We are moving closer to this goal with new breakthroughs in deep learning, but methods successful in analysing this data for prediction of future diseases and risk are not performing better than methods doctors use, even though doctors’s predictions are based on their expert knowledge and experience from a tiny subset of all patients in a database.

We will work on applying deep learning for Electronic Health Records (EHR). More specifically, we will look at neural network architectures from the Deepr and Doctor AI articles for predicting future diagnoses, e.g., the term (birth week) of 90 000 pregnant women or the risk of evolving type 2 diabetes.

The project is done in cooperation with Pers Lab (under Center for Basic Metabolic Research at University of Copenhagen and State Serum Institute (SSI)) as well as Ole Winther from the Cognitive Systems research section at DTU Compute.

P. Nguyen et al.: “Deepr: A Convolutional Net for Medical Records”, July 2016.

E. Choi et al.: “Doctor AI: Predicting Clinical Events via Recurrent Neural Networks”, November 2015.
